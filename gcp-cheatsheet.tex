%%%%%%%%%%%%%%%%%%%%%%%%%%%%%%%%%%%%%%%%%%%%%%%%%%%%%%%
% GCP Professional Cloud Architect Cheat Sheet
%
% Edited by Aleksandra T. Sekalska
%
%%%%%%%%%%%%%%%%%%%%%%%%%%%%%%%%%%%%%%%%%%%%%%%%%%%%%%%

% \documentclass{article}
\documentclass[9pt]{extarticle}
\usepackage[landscape]{geometry}
\usepackage{url}
\usepackage{multicol}
\usepackage{amsmath}
\usepackage{amsfonts}
\usepackage{tikz}
\usetikzlibrary{decorations.pathmorphing}
\usepackage{amsmath,amssymb}
\usepackage{tabularx}

\usepackage{colortbl}
\usepackage{xcolor}
\usepackage{mathtools}
\usepackage{amsmath,amssymb}
\usepackage{enumitem}
\usepackage{tabto}
\usepackage{enumitem}
\usepackage{graphicx}
\usepackage{tabu}
\usepackage{fontawesome5}
\usepackage{hyperref}

\title{GCP Professional Cloud Architect}

\advance\topmargin-.8in
\advance\textheight3in
\advance\textwidth3in
\advance\oddsidemargin-1.5in
\advance\evensidemargin-1.5in
\parindent0pt
\parskip2pt
\newcommand{\hr}{\centerline{\rule{3.5in}{1pt}}}
%\colorbox[HTML]{e4e4e4}{\makebox[\textwidth-2\fboxsep][l]{texto}
\begin{document}

\begin{center}{\huge{\textbf{Google Cloud Architect Cheatsheet}}}\\
{\large Compiled by Aleksandra T. Sekalska }\\
{\normalsize Last Updated December 18, 2019}
\end{center}
\begin{multicols*}{3}

\tikzstyle{mybox} = [draw=black, fill=white, very thick, rectangle, rounded corners, inner sep=10pt, inner ysep=10pt]
\tikzstyle{fancytitle} =[fill=black, text=white, font=\bfseries]

%\faTelegram  
%------------ What is GCP Professional Cloud Architect Certificate  ---------------
\begin{tikzpicture}
\node [mybox] (box){%
    \begin{minipage}{0.3\textwidth}
    What is GCP Professional Cloud Architect Certificate\\
	 \\
        
    \setlist{nolistsep}	
    \begin{itemize}
    	\item Design and plan a cloud solution architecture
		\item Manage and provision the cloud solution infrastructure
        \item Design for secure and compliance
        \item Analyze and optimize technical and business processes
        \item Manage implementations of cloud architecture
        \item Ensure solution and operations reliability
    \end{itemize}
    
    \end{minipage}
};
\node[fancytitle, right=10pt] at (box.north west) {GCP Cloud Architect};
\end{tikzpicture}



%------------ VIRTUAL MACHINES ---------------
\begin{tikzpicture}
\node [mybox] (box){%
    \begin{minipage}{0.3\textwidth}
    \setlist{nolistsep}	
    Google Cloud VPC provides networking functionality to Compute Engine virtual machine instances, Google Kubernetes Engine containers, and the App Engine flexible environment. \\
    {\color{blue} \textbf{Private Cloud Networks:}}\\
    	A Private Cloud Network is a virtual version of a physical network, such as a data center network. Projects can contain multiple VPC networks. \\
        VPC networks, including their associated routes and firewall rules, are global resources. They are not associated with any particular region or zone.\\
        Subnets are regional resources. Each subnet defines a range of IP addresses. \\
        Virtual Machines (VM) running in Google's global data center. Ideal for when you need complete control over your infrastructure and direct access to high-performance hardward or need OS-level changes. \\
        Google Cloud VPC networks are global; subnets are regional (coursera)\\
        You control the topology of your VPC network:
        \begin{itemize}
            \item Use its route table to forward traffic within the network, even across subnets.
            \item Use its firewall to control what network traffic is allowed.
            \item You can use VPC Peering to interconnect different VPC across GCP.
            \item 
        \end{itemize}    
    {\color{blue} \textbf{Compute Engine:}}\\
        It offers managed virtual machines. Pick memory and CPU: use predefined types, or make a custom VM; (coursera) \\
        You can choose 2 of persistant storage: SD or standard. (coursera) \\
        You can choose preemtible instances: high throughput to storage at no extra cost; no longer than 24 hours \\
    \end{minipage}

};
\node[fancytitle, right=10pt] at (box.north west) {Virtual Machines};
\end{tikzpicture}





%------------ STORAGE ---------------
\begin{tikzpicture}
\node [mybox] (box){%
    \begin{minipage}{0.3\textwidth}
    Overview\\
    {\color{blue} \textbf{Cloud Storage:}}\\ 
        It is a binary large-object storage (coursera) \\
        Always encrypt the data on the server side (coursera) \\
        The files are organized into buckets: the buckets have globally unique name (coursera) \\
        Choose among Cloud Storage classes: \\
        \begin{itemize}
            \item \textbf{Multi-regional}: Most frequenlty accessed\\
            \item \textbf{Regional}: Accessed frequently within a region\\
            \item \textbf{Nearline}: Accessed less than once a month\\
            \item \textbf{Coldline}: Accessed less than once a year\\
        \end{itemize}    
        There are several ways to bring data to Cloud Storage:
        \begin{itemize}
            \item \textbf{Online Transfer}:
            \item \textbf{Storage Transfer Service}:
            \item \textbf{Transfer Appliance}:
        \end{itemize}    
    {\color{blue} \textbf{Cloud BigTable:}}\\
        To be edited \\
    {\color{blue} \textbf{Cloud SQL:}}\\ 
        To be edited \\
    {\color{blue} \textbf{Cloud Spanner:}}\\
        To be edited \\
    {\color{blue} \textbf{Cloud DataStore:}}\\ 
        To be edited \\
    {\color{blue} \textbf{Cloud Memorystore:}}\\ 
        To be edited \\
    \end{minipage};
};
\node[fancytitle, right=10pt] at (box.north west) {Storage};
\end{tikzpicture}


% ------------ CONTAINERS -----------------
\begin{tikzpicture}
\node [mybox] (box){%
    \begin{minipage}{0.3\textwidth}
    \setlist{nolistsep}	
    Overview \\
    {\color{blue} \textbf{Containers:}}\\ 
        Choosing an option to run containers \\
    {\color{blue} \textbf{Kubernetes:}}\\ 
        Owner (full access to resources, manage roles), Editor (edit access to resources, change or add), Viewer (read access to resources)\\
    {\color{blue} \textbf{Kubernetes Compute Engine:}}\\ 
        Owner (full access to resources, manage roles), Editor (edit access to resources, change or add), Viewer (read access to resources)\\
    

    \end{minipage}
};
\node[fancytitle, right=10pt] at (box.north west) {Containers};
\end{tikzpicture}


% ------------ APPLICATIONS -----------------
\begin{tikzpicture}
\node [mybox] (box){%
    \begin{minipage}{0.3\textwidth}
    \setlist{nolistsep}	
    GCP's monitoring, logging, and diagnostics solution. Provides insights to health, performance, and availability of applications.\\
    Main Functions \\
    {\color{blue} \textbf{App Engine:}}\\ 
        Owner (full access to resources, manage roles), Editor (edit access to resources, change or add), Viewer (read access to resources)\\
    {\color{blue} \textbf{Cloud Endpoints:}} \\
        Owner (full access to resources, manage roles), Editor (edit access to resources, change or add), Viewer (read access to resources)\\
    {\color{blue} \textbf{Apigee Sense:}}\\ 
        Owner (full access to resources, manage roles), Editor (edit access to resources, change or add), Viewer (read access to resources)\\
    
   
    \end{minipage}
};
\node[fancytitle, right=10pt] at (box.north west) {Applications};
\end{tikzpicture}


% ------------ DEVELOPING, DEPLOYING AND MONITORING -----------------
\begin{tikzpicture}
\node [mybox] (box){%
    \begin{minipage}{0.3\textwidth}
    \setlist{nolistsep}	
    Overview \\
    {\color{blue} \textbf{Cloud Source Repositories:}}\\
    {\color{blue} \textbf{Cloud Functions:}}\\
    
    \end{minipage}
};
\node[fancytitle, right=10pt] at (box.north west) {Developing, Deploying and Monitoring????};
\end{tikzpicture}


% ------------ BIG DATA AND ML -----------------
\begin{tikzpicture}
\node [mybox] (box){%
    \begin{minipage}{0.3\textwidth}
    Overview\\
    {\color{blue} \textbf{Cloud Dataflow:}}\\
    {\color{blue} \textbf{BigQuery:}}\\
    {\color{blue} \textbf{Cloud Pub/Sub:}}\\
    {\color{blue} \textbf{Cloud Datalab:}}\\
    {\color{blue} \textbf{GCP Machine Learning Services:}}\\
    
    \setlist{nolistsep}	
    \end{minipage}
};
\node[fancytitle, right=10pt] at (box.north west) {Compute Choices};
\end{tikzpicture}



% ------------ CLOUD IAM Part I-----------------
\begin{tikzpicture}
\node [mybox] (box){%
    \begin{minipage}{0.3\textwidth} 
    \setlist{nolistsep}	
    Cloud Identity and Access Management (IAM) lets administrators authorize who can take action on specific resources, giving you full control and visibility to manage cloud resources centrally. For established enterprises with complex organizational structures, hundreds of workgroups, and potentially many more projects, Cloud IAM provides a unified view into security policy across your entire organization, with built-in auditing to ease compliance processes. \\
    It is the process of determining who can do what on which resource. \\
    With Cloud IAM you manage access control by defining who (identity) has what access (role) for which resource. In Cloud IAM, permission to access a resource isn't granted directly to the end user. Permissions are grouped into roles, and roles are granted to authenticated members. A Cloud IAM policy defines and enforces what roles are granted to which members and this policy is attached to a resource.\\
    All the components are resources: \\
    organizations, projects, folders, services. Those resources are organized hierarchically: \\
    %\includegraphics[width=8.45cm, height=9.7cm]{figures/cloud-folders-hierarchy.png}\\
    The organization is the root node in the hierarchy. Folders are children of the organization. Projects are children of the organization, or of a folder. Resources for each service are descendants of projects. \\
    \end{minipage}
};
\node[fancytitle, right=10pt] at (box.north west) {GCP IAM Part I};
\end{tikzpicture}

% ------------ CLOUD IAM Part II -----------------
\begin{tikzpicture}
\node [mybox] (box){%
    \begin{minipage}{0.3\textwidth} 
    \setlist{nolistsep} 
    {\color{blue} \textbf{\href {https://cloud.google.com/iam/docs/understanding-roles}{Roles}:}}\\ 
        A collection of permissions.\\
        Permissions determine what specific operations are allowed on resource. When you grant a role to a member, you grant all the permissions that the role contains. \\ 
        Permissions are not directly assigned to a Member. Permissions are assigned to a Role and that role is assigned to Members.\\
         \begin{itemize}
            \item \textbf{Primitive Roles:} These roles are \textbf{Owner}, \textbf{Editor} and \textbf{Viewer}. Avoid using these roles if possible, because they include a wide range of permissions accross all Google Cloud services.  
            \item \textbf{Predefined Roles:} Roles that give finer-grained access control than the primitive roles. ???
            \item \textbf{Custom Roles:} Roles that you create to tailor permissions to the needs of your organization when predefined roles don't meet your needs.
        \end{itemize}
    {\color{blue} \textbf{Members:}} \\
        A member can be a person, a service account, a Google Group, or a G Suite or Cloud Identity domain. It is represented by an email address. \\
    {\color{blue} \textbf{\href {https://cloud.google.com/iam/docs/service-accounts}{Service Accounts}:}} \\
        It is represented by an email address and it is associated with an application or a server, not a person. Applications use service accounts to make authorized API calls.\\
    {\color{blue} \textbf{Policy:}} \\
        The Cloud IAM policy binds one or more members to a role. When you want to define who has what type of access on a resource, you create a policy and attach it to the resource. \\
    More info \href{https://cloud.google.com/iam/docs/overview}{here.}    
    \end{minipage}
};
\node[fancytitle, right=10pt] at (box.north west) {GCP IAM Part II};
\end{tikzpicture}

% ------------ Resource Monitoring -----------------
\begin{tikzpicture}
\node [mybox] (box){%
    \begin{minipage}{0.3\textwidth}
    \setlist{nolistsep}	
    Aggregates metrics, logs and events for monitoring, logging and tracking diagnostics.\\
    {\color{blue} \textbf{\href {https://cloud.google.com/logging/docs/basic-concepts}{Stackdriver Logging}:}} \\
        Allows you to store, search, analyze, monitor, and alert on log data and events from GCP and AWS. \\
    {\color{blue} \textbf{\href {https://cloud.google.com/monitoring/docs/concepts}{Stackdriver Monitoring}:}} \\
        Provides visibility into the performance, uptime and overall health of cloud-powered applications. It collects metrics, events, and metadata from different services. \\
        Stackdriver ingests data and generates dashboards, charts, and alerts.\\
    {\color{blue} \textbf{\href {https://cloud.google.com/error-reporting/docs/}{Stackdriver Error Reporting}:}} \\
        Counts, analyzes, and aggregates the crashed running cloud services.  \\
    {\color{blue} \textbf{\href {https://cloud.google.com/trace/docs/overview}{Stackdriver Trace}:}}\\
        A distributed tracing system for GCP that collects latency data from App Engine applications and displays it in near real time.\\
    {\color{blue} \textbf{\href {https://cloud.google.com/debugger/docs/}{Stackdriver Debugger}:}} \\
        Lets you inspect the state of an application, at any code location, without stopping or slowing down the running app.\\
    {\color{blue} \textbf{\href {https://cloud.google.com/profiler/docs/concepts-profiling}{Stackdriver Profiler}:}} \\
        A statistical profiler. It does not require pervasive changes to the program code to collect data. Instead, a piece of code, called profiling agent, is essentially attached to the code, where it can periodically look at the call stack of the program to collect information about, for example, CPU or memory usage. \\
      
    
    \end{minipage}
};
\node[fancytitle, right=10pt] at (box.north west) {Cloud Stackdriver};
\end{tikzpicture}

% ------------ Interconnecting Networks -----------------
\begin{tikzpicture}
\node [mybox] (box){%
    \begin{minipage}{0.3\textwidth}
    \setlist{nolistsep}	
    Overview\\
    {\color{blue} \textbf{Cloud VPN:}}\\ Some text to be introduced \\
    {\color{blue} \textbf{Cloud Interconnect:}} \\
        Some text to be introduced \\
    {\color{blue} \textbf{Cloud Peering:}} \\
        Some text to be introduced \\
    {\color{blue} \textbf{Shared VPC and VPC Peering:}}\\ 
        Some text to be introduced \\
    
    \end{minipage}
};
\node[fancytitle, right=10pt] at (box.north west) {Interconnecting Networking};
\end{tikzpicture}


% ------------ Load Balancing and Autoscaling -----------------
\begin{tikzpicture}
\node [mybox] (box){%
    \begin{minipage}{0.3\textwidth}
    \setlist{nolistsep}	
    Choosing a Load Balancer \\
    {\color{blue} \textbf{HTTP(S) Load Balancing:}} \\
        Some text to be introduced \\
    {\color{blue} \textbf{SSL Proxy Load Balancing:}} \\
        Some text to be introduced \\
    {\color{blue} \textbf{TCP Proxy Load Balancing:}}\\ 
        Some text to be introduced \\
    {\color{blue} \textbf{Network Load Balancing:}}\\ 
        Some text to be introduced \\
    {\color{blue} \textbf{Internal Load Balancing:}} \\
        Some text to be introduced \\
    	  Intro
	    \begin{itemize}
	    	\item \textbf{TCP/UDP Load Balancing:} Tables partitioned based on the data's ingestion (load) date or arrival date. Each partitioned table will have  pseudocolumn \_PARTITIONTIME, or time data was loaded into table. Pseudocolumns are reserved for the table and cannot be used by the user. 
			\item \textbf{Internal HTTP(s) Load Balancing:} Tables that are partitioned based on a TIMESTAMP or DATE column.\\
		\end{itemize}
    

    \end{minipage}
};
\node[fancytitle, right=10pt] at (box.north west) {Load Balancing and Autoscaling};
\end{tikzpicture}




% ------------ Infrastructure Automation -----------------
\begin{tikzpicture}
\node [mybox] (box){%
    \begin{minipage}{0.3\textwidth}
    	{\color{blue} \textbf{Deployment Manager:}}\\ 
            Some text to be introduced \\
    	\textbf{Explore best practices:}\\
        {\color{blue} \textbf{GCP Marketplace:}} Some text to be introduced \\
    
    \end{minipage}
};
\node[fancytitle, right=10pt] at (box.north west) {Infrastructure Automation};
\end{tikzpicture}


% ------------ Managed Services -----------------
\begin{tikzpicture}
\node [mybox] (box){%
    \begin{minipage}{0.3\textwidth}
    \setlist{nolistsep}	
    	{\color{blue} \textbf{BigQuery:}}\\ 
            Some text to be introduced \\
    	{\color{blue} \textbf{Cloud Dataflow:}} \\
            Some text to be introduced \\
    	{\color{blue} \textbf{Cloud Dataprep:}}\\ 
            Some text to be introduced \\
    	{\color{blue} \textbf{Cloud Dataproc:}} \\
            Some text to be introduced \\
    	
    
    \end{minipage}
};
\node[fancytitle, right=10pt] at (box.north west) {Managed Services};
\end{tikzpicture}


% ------------ Case Studies -----------------
\begin{tikzpicture}
\node [mybox] (box){%
    \begin{minipage}{0.3\textwidth}
    \setlist{nolistsep}	
    Overview
    {\color{blue} \textbf{Mountkirk Games:}}\\
    {\color{blue} \textbf{Dress4Win:}}\\
    {\color{blue} \textbf{TerramEarth:}}\\
    
    \end{minipage}
};
\node[fancytitle, right=10pt] at (box.north west) {Case Studies};
\end{tikzpicture}


\end{multicols*}
\end{document}
